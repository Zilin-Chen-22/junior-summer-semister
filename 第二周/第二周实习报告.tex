\documentclass[11pt]{article}
\usepackage{ctex}
\usepackage{graphicx}
\usepackage{amsmath}
\usepackage{float}
\usepackage[a4paper,left=25mm,right=25mm,top=31mm,bottom=31mm]{geometry}
\graphicspath{{./images/}}

\begin{document}

\begin{center}
    \LARGE \textbf{企业实习\;第二周实习报告}\\
    \vspace{10pt}
    \normalsize 华海清科CMP事业部II部边缘抛光部\;\;陈子林
\end{center}

\section{本周工作主要内容}
本周为实习第二周(6.30-7.04),主要工作内容如下:
\begin{enumerate}
    \item 利用python搭建边缘抛光系统仿真器平台,进行运动学仿真,抛光相关参数(如MRR)需手动输入。目前能够支持自动进行运动学仿真,在给定MRR时计算出相应的抛光效果,生成效果图以及动态过程展示。
    \item 优化算法,加速仿真过程(初代计算时间需要10小时以上,优化后仅需10min以内)
    \item 进行进入洁净间培训,深入了解工艺过程
    \item 设计过定位抛光垫调平衡方案,优化调整流程
\end{enumerate}

\section{后续工作计划}
后续工作计划如下:
\begin{enumerate}
    \item 添加自动计算MRR部分
    \item 进行物理实验,研究抛光参数对MRR的具体影响
    \item 对比仿真结果与实际实验结果,验证仿真器的准确性
\end{enumerate}

% \newpage
% \begin{thebibliography}{99}
%     \item D. Zhao, Y. He, T. Wang and X. Lu, "Effect of Kinematic Parameters and Their Coupling Relationships on Global Uniformity of Chemical-Mechanical Polishing," in IEEE Transactions on Semiconductor Manufacturing, vol. 25, no. 3, pp. 502-510, Aug. 2012, doi: 10.1109/TSM.2012.2190432.
% \end{thebibliography}

\end{document}