\documentclass{ctexart}
\usepackage{amsmath}
\usepackage{float}
\usepackage{graphicx}
\usepackage[a4paper,left=25mm,right=25mm,top=31mm,bottom=31mm]{geometry}
\pagestyle{plain}
\graphicspath{{./images}}


\begin{document}

\begin{center}
    \LARGE \textbf{企业实习\;第七周实习报告}\\
    \vspace{10pt}
    \normalsize 小米自动驾驶与机器人部\;\;陈子林
\end{center}

\section{本周工作主要内容}

本周为小米实习第四周 (8.04--8.08),主要工作内容如下:

\begin{enumerate}
    \item 对规划的机械臂moveL路径进行滤波处理,得到平滑的关节曲线。针对原始路径存在的抖动和不连续问题,尝试了多种滤波算法,并对比了滤波前后的轨迹效果,确保机械臂运动更加平稳可靠。
    \item 运动学逆解器带入宇树机器人h1\_2的urdf文件对算法进行验证,证明算法普适性和鲁棒性
    \item 配置实验室电脑及本地环境,使自己能够远程ssh连接实验室电脑并读取h1\_2机器人的数据,或对其进行远程操控开发。期间解决了部分依赖库安装和网络权限问题,提升了开发效率。
    \item 进行宇树h1\_2机器人上肢重复运动精度测试,记录技术文档,准备与m92u机器人进行对比测试。测试过程中详细记录了各项参数和实验现象,为后续数据分析和对比提供了基础。
\end{enumerate}

\section{后续工作计划}

针对目前实习目标和工作进度,下一阶段计划如下:

\begin{enumerate}
    \item 连接m92u机器人进行物理实验,进行上肢重复运动精度测试,对比宇树h1\_2机器人。计划采用相同的测试流程和评价指标,确保对比结果的客观性和可复现性。
    \item 完成两个机器人的GBT 12642--2013/ISO 9283:1998测试。将根据标准流程进行实验,收集相关数据,并对测试结果进行整理和分析。
    \item 为宇树h1\_2搭建仿真系统,对比仿真结果与物理实验结果并优化。后续将根据对比结果调整仿真参数,提升仿真系统的准确性和实用性。
\end{enumerate}

\end{document}