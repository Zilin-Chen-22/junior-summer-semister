\documentclass{ctexart}
\usepackage{amsmath}
\usepackage{float}
\usepackage[a4paper,left=25mm,right=25mm,top=31mm,bottom=31mm]{geometry}
\pagestyle{plain}


\begin{document}

\begin{center}
    \LARGE \textbf{企业实习\;第四周实习报告}\\
    \vspace{10pt}
    \normalsize 小米自动驾驶与机器人部\;\;陈子林
\end{center}

\section{本周工作主要内容}

本周为小米实习第一周 (7.14--7.18),主要工作内容如下:

\begin{enumerate}
    \item 完成入职手续,包括签订实习协议和网络安全承诺书,系统学习了部门规章制度、数据保密要求以及代码仓库的访问流程。同时,通过部门介绍初步了解了自动驾驶与机器人部的组织结构和核心研究方向。

    \item 与导师和所在小组成员进行了初步沟通,明确了本阶段的实习任务目标。本次实习主要聚焦于人形机器人中七自由度机械臂的运动规划算法开发。为了确保后续工作具有方向性,查阅了团队内前期项目资料,并调研了相关国内外研究动态。

    \item 在导师指导下,初步熟悉了机械臂控制与轨迹规划所需的软件技术栈,重点了解了如 MoveIt(基于 ROS 的运动规划框架)、Pinocchio(动力学建模库)及强化学习在机器人路径优化中的应用思路。通过阅读官方文档和开源项目,建立了对整体系统的初步认知。

    \item 配置了实习所需的软件开发环境,与小组成员统一系统配置(Ubuntu 22.04 + ROS2 Humble + Conda + VSCode 等),解决了依赖冲突、Python 包版本不一致等问题,确保后续协作顺利进行。

    \item 学习了机器人仿真相关工具,尤其是 MuJoCo 引擎的基本使用方法。尝试使用官方 demo 加载机械臂模型,了解其物理仿真流程,为后续将算法部署至仿真环境打下基础。
\end{enumerate}

\section{后续工作计划}

针对目前实习目标和工作进度,下一阶段计划如下:

\begin{enumerate}
    \item 在已有开源项目基础上,搭建简化的七自由度机械臂轨迹规划流程,先在静态环境中实现基本的运动路径生成,并与组内仿真框架进行初步对接,实现仿真中的可视化验证。

    \item 深入研究轨迹规划相关算法,包括逆运动学求解(IK)、运动路径插值与避障策略,逐步提高规划精度与计算效率,并尝试在小规模数据集上进行训练与调试。

    \item 继续完善 Mujoco 仿真环境的部署与使用,计划实现机器人在仿真环境中的动作执行与实时反馈,进一步验证运动算法的有效性、稳定性与鲁棒性,并准备与实际控制系统接轨的初步接口。
\end{enumerate}


\end{document}