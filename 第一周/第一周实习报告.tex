\documentclass[11pt]{article}
\usepackage{ctex}
\usepackage{graphicx}
\usepackage{amsmath}
\usepackage{float}
\usepackage[a4paper,left=25mm,right=25mm,top=31mm,bottom=31mm]{geometry}
\graphicspath{{./images/}}

\begin{document}

\begin{center}
    \LARGE \textbf{企业实习\;第一周实习报告}\\
    \vspace{10pt}
    \normalsize 华海清科CMP事业部II部边缘抛光部\;\;陈子林
\end{center}

\section{本周工作主要内容}
本周为实习第一周(6.23-6.27),主要工作内容如下:
\begin{enumerate}
    \item 签订实习协议,了解实习单位的基本情况。华海清科CMP事业部II部边缘抛光部主要负责半导体晶圆的边缘抛光工艺。
    \item 学习边缘抛光机的工作原理和操作流程。边缘抛光机主要用于半导体晶圆的边缘处理,以提高晶圆的质量和良率。
    \item 阅读相关文献,了解边缘抛光的技术背景和发展现状。通过文献学习,掌握了边缘抛光的基本原理和常用方法。
    \item 与组内成员进行讨论,了解当前边缘抛光工艺中存在的问题和挑战,明确后续工作方向。
    \item 参观生产车间,观察实际生产中的边缘抛光过程,了解设备的运行状态和维护要求。
    \item 阅读相关书籍,查阅资料,了解芯片生产全过程,以及先进封装技术,为后续工作开展做好基础学习。
\end{enumerate}

\section{后续工作计划}
后续工作计划如下:
\begin{enumerate}
    \item 进行进入洁净间的相关培训,为后在洁净室内的工作做好准备。
    \item 深入学习边缘抛光机的操作技能,掌握设备的调试和维护方法。
    \item 开展边缘抛光工艺的优化研究,重点关注抛光性能与各项抛光可调参数之间的关系。
    \item 收集和分析边缘抛光过程中产生的数据,为后续搭建仿真平台做好准备。
    \item 继续阅读相关文献,了解国内外在边缘抛光领域的最新研究进展,特别是与化学机械抛光(CMP)相关的技术。
\end{enumerate}

\newpage
\begin{thebibliography}{99}
    \item D. Zhao, Y. He, T. Wang and X. Lu, "Effect of Kinematic Parameters and Their Coupling Relationships on Global Uniformity of Chemical-Mechanical Polishing," in IEEE Transactions on Semiconductor Manufacturing, vol. 25, no. 3, pp. 502-510, Aug. 2012, doi: 10.1109/TSM.2012.2190432.
    \item Zhao D, Wang T, He Y, et al. Kinematic optimization for chemical mechanical polishing based on statistical analysis of particle trajectories[J]. IEEE transactions on semiconductor manufacturing, 2013, 26(4): 556-563.
    \item Aoki T, Hirasawa M, Izunome K, et al. Development of Novel Bevel Profile for Wafer-level Stacking Technology[C]//2021 International Conference on Electronics Packaging (ICEP). IEEE, 2021: 123-124.
    \item Kobayashi N, Wu Y, Nomura M, et al. Precision treatment of silicon wafer edge utilizing ultrasonically assisted polishing technique[J]. journal of materials processing technology, 2008, 201(1-3): 531-535.
    \item  杨发顺 集成电路芯片制造[M]. 北京: 清华大学出版社, 2018. 
    \item  姚玉 芯片先进封装制造[M]. 广州: 广州暨南大学出版社, 2019. 
    \item 周玉刚、张荣.微电子封装技术:面向新工科的电工电子信息基础课程系列教材[M].北京:清华大学出版社,2023.
\end{thebibliography}

\end{document}