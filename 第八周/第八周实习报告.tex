\documentclass{ctexart}
\usepackage{amsmath}
\usepackage{float}
\usepackage{graphicx}
\usepackage[a4paper,left=25mm,right=25mm,top=31mm,bottom=31mm]{geometry}
\pagestyle{plain}
\graphicspath{{./images}}

\begin{document}

\begin{center}
    \LARGE \textbf{企业实习\;第八周实习报告}\\
    \vspace{10pt}
    \normalsize 小米自动驾驶与机器人部\;\;陈子林
\end{center}

\section{本周工作主要内容}

本周为小米实习第四周 (8.11--8.15)。由于m92u原型机的生产调试进度略有延迟,本周工作重心仍围绕现有的宇树H1\_2机器人平台展开,主要完成了以下四项核心任务:

\begin{enumerate}
    \item \textbf{完成宇树H1\_2机器人的GBT 12642--2013/ISO 9283:1998标准性能测试:}
          严格遵循工业机器人性能测试规范,对H1\_2机器人进行了包括位姿准确度与重复性等指标的测量。利用高精度激光跟踪仪或运动捕捉系统采集末端执行器在笛卡尔空间的实际位姿数据,并通过脚本对原始数据进行处理和分析,生成详细的测试报告,为后续与m92u的性能对标奠定数据基础。(可能涉及企业保密问题,暂无可用图片及详细数据)

    \item \textbf{优化通用逆运动学(IK)求解器接口模块:}
          针对不同机器人平台测试的需求,对现有的逆解器调用模块进行了通用化改造和接口优化。主要工作包括:设计标准化的配置文件格式,只需提供目标机器人的URDF模型文件,模块即可自动适配;开发了用户友好的命令行工具和简单的可视化界面,方便测试人员快速设定目标位姿、调用逆解并验证结果的合理性。

    \item \textbf{进行宇树H1\_2机器人负重耐久测试规划与调试:}
          设计了一个有负载、循环应力的测试动作序列:机器人需在负重状态下反复执行“深蹲、起立”的循环动作。完成了该动作序列在机器人控制系统内部的轨迹规划和参数配置。但由于机器人右踝关节在其他测试中意外断裂,导致无法进行落地负重测试,预计将在下周完成修复或更换。

    \item \textbf{构建基于MuJoCo的机器人动力学仿真环境:}
          启动了面向m92u和H1\_2的MuJoCo仿真平台搭建工作。本周主要完成了:将H1\_2的URDF模型成功导入MuJoCo,搭建了包含基本地面和重力场的仿真场景,进行了初步的关节角度输入控制测试,验证了模型在仿真环境中的基本运动能力。
\end{enumerate}

\section{后续工作计划}

基于当前项目进展和遇到的实际情况,下一阶段的工作将聚焦于以下关键目标:

\begin{enumerate}
    \item \textbf{优先完成m92u的基准性能与耐久测试:} 一旦m92u原型机达到可运行状态,将立即着手执行与h1\_2相同的GBT/ISO标准性能测试流程(位姿、轨迹精度等),以及规划好的负重耐久循环测试。获取m92u的第一手性能数据是进行对标分析的核心前提。
    \item \textbf{深度优化MuJoCo仿真环境:} 在现有基础仿真模型上,重点提升仿真的真实性和实用性:根据实物测量数据校准仿真模型动力学参数。目标是让仿真结果能有效指导现实中的参数调试、算法验证和潜在问题的预判。
\end{enumerate}

\section{反馈问题回答}
\noindent
\textbf{问题:} 两个机器人性能对比的目的?\\
\textbf{回答:} 进行H1\_2与m92u的详尽性能对标测试,目的在于行业技术对标与自研方案可行性验证。通过对比两者在相同标准测试项目下的表现,客观评估m92u所采用的本体构型在实现目标性能指标上的优势和不足;测试结果将直接反映m92u关节驱动单元、底层伺服控制算法的实际效能及其与行业成熟产品的差距;基于对比分析结果,识别m92u当前设计或控制中的瓶颈环节,向结构设计、硬件选型和软件控制团队提供具体、量化的改进建议,推动下一代原型机或最终产品的性能提升。

\end{document}